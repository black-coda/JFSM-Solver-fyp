\chapter{Methodology}

\section{Introduction}
The methodology section serves as the backbone of this project, providing a comprehensive understanding of the algorithms approach used to analyze linear multistep methods (LMMs) and solve ordinary differential equation (ODE). It outlines the systematic procedures, techniques, and tools utilized throughout the solver-project lifecycle, shedding light on the intricacies of our analysis and solution methodology.This section plays a pivotal role in elucidating how we approached the analysis of LMMs and their application in solving ODE questions. It provides clarity on the selection of LMMs, the formulation of numerical algorithms, the validation of results, and the integration of computational techniques into a cohesive framework.


\subsection*{Flutter as a Development tool}
A notable aspect of this solver is the utilization of Flutter, Google's open-source UI software development kit, for building the desktop application. Flutter was chosen as the development framework due to its versatility and efficiency in creating cross-platform applications that run seamlessly on various operating systems, including Windows, macOS, and Linux.

The decision to adopt Flutter stems from its numerous advantages for desktop app development. Firstly, Flutter offers a single codebase that can be used to target multiple platforms, eliminating the need to maintain separate codebases for different operating systems. This not only streamlines the development process but also ensures consistency in the user experience across platforms and it also implies that the solver will run any of the following operating system ranging from the IOS for mobile users to Windows for desktop to users, but currently we will be sticking to only desktop apps preferably the Windows operating system.

Additionally, Flutter provides a rich set of widgets and tools for designing visually appealing and interactive user interfaces. The flexibility of Flutter's UI framework allows for the creation of custom UI components tailored to the specific requirements of our desktop application. This is particularly advantageous for visualizing numerical data and facilitating user interactions with the ODE solver.

Furthermore, Flutter boasts excellent performance characteristics, thanks to its high-performance rendering engine, Dart language optimization, and ahead-of-time compilation. This ensures smooth and responsive user experiences, even when performing complex numerical computations within the application.

By leveraging Flutter for desktop app development and also Dart(\textit{flutter is written in dart}), we aim to deliver a robust and user-friendly application that combines the power of LMM analysis with intuitive UI design. The following sections will delve deeper into the methodology employed, including the design considerations, integration of LMM algorithms, validation techniques, and deployment strategies.



The development of the solver is divided into two modules, the first module which involves the development of the algorithms and UI for the analysis of the linear multistep method, and the second module which involves using the method to solve a particular problem. 

% The formulation of numerical algorithms is a critical step in our methodology. We utilize the general k-step LMM, which involves a linear combination of previous points and derivative values to solve first-order ODEs. This approach allows us to leverage the efficiency of multistep methods, which refer to several previous points and derivative values, thereby gaining efficiency by keeping and using the information from previous steps rather than discarding it, which are also used in solving stiff problems.


\subsection{Module 1: Analysis of Linear multistep method}
The general $k-step$ linear multistep method takes the form 


\begin{math}
   y_{n+k} + \alpha_{k-1}y_{n+k-1}+ \dots + \alpha_0x_n = h(\beta_kf_{n+k}+ \beta_{k-1}f_{n+k-1}+ \dots + \beta_0f_n) 
\end{math}

which is equal to 
\begin{equation}
   \sum_{j=0}^{k} \alpha_j y_{n+j} = h \sum_{j=0}^{k} \beta_j f(x_{n+j}, y_{n+j})
\end{equation} \cite{2022JFatokunEtAl}

The properties such as \textbf{Consistency},\textbf{Zero Stability},\textbf{Convergence} are investigated, also the \textbf{Error constant} and \textbf{Order} is also calculated by the software.

\subsubsection{Consistency}




\[
c_p = \sum \left( \frac{(jp)^!}{\alpha_j} - \frac{(p(j-p-1))!}{\beta_j} \right).
\]