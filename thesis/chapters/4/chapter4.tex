\chapter{Numerical Results}
\section*{Illustrative Numerical Examples Demonstrating Solver Performance}

In this section, we provide numerical examples to offer a comprehensive illustration of the accuracy and functionality of the proposed solver. These examples serve to demonstrate the solver's capabilities across various scenarios and shed light on its performance under different conditions. We carefully select three main examples that effectively showcase the solver's effectiveness and robustness in solving differential equations encountered in practical applications. Through these examples, we aim to provide insights into the solver's behavior, its accuracy in approximating solutions, and its suitability for diverse problem types.

This will also be done as based on the module,


\section{Module 1: Analysis of Linear multistep}
\subsection{The Quade's Method}
Consider the Quade's method of the form \cite{lambert1977}

\begin{equation}
    y_{n+4} - \frac{8}{19}y_{n+3} + \frac{8}{19}y_{n+1} - y_{n} =  \frac{6}{19}h\bigl(f_{n+4}+4f_{n+3}+4f_{n+1}+f_{n}\bigr)
\end{equation}


from the above equation, we can see that the method is a 4-step method,


where:
\[
\begin{aligned}&\alpha_0 &= -1, \alpha_1 = \frac{8}{19}, \alpha_2 = 0, \alpha_3 = -\frac{8}{19}, \alpha_4 = 1 \\
&\beta_0 &= \frac{6}{19}, \beta_1 = \frac{24}{19}, \beta_2 = 0, \beta_3 = \frac{24}{19}, \beta_4 = \frac{6}{19} 
\end{aligned}
\]

in other to determine the order of the Quade's method, we use (3.4), we obtain the following values:


\begin{eqnarray}
    c_0 = \sum_{i=0}^{4}(\alpha_i) = \frac{8}{19} - \frac{8}{19} = 0 \\
    c_1 = \sum_{i=0}^{4}(i\alpha_i - \beta_i) = \frac{60}{19} - \frac{60}{19} = 0 \\
    c_2 = \sum_{i=0}^{4}(\frac{i^2}{2!} \alpha_i - i \beta_i) = \frac{120}{19} - \frac{120}{19} = 0 \\
    c_3 = \sum_{i=0}^{4}(\frac{i^3}{3!} \alpha_i - \frac{i^2}{2!} \beta_i) = \frac{168}{19} - \frac{168}{19} = 0 \\
    c_4 = \sum_{i=0}^{4}(\frac{i^4}{4!} \alpha_i - \frac{i^3}{3!} \beta_i) = \frac{176}{19} - \frac{176}{19} = 0 \\
    c_5 = \sum_{i=0}^{4}(\frac{i^5}{5!} \alpha_i - \frac{i^4}{4!} \beta_i) = \frac{146}{19} - \frac{146}{19} = 0 \\
    c_6 = \sum_{i=0}^{4}(\frac{i^6}{6!} \alpha_i - \frac{i^5}{5!} \beta_i) = \frac{100}{19} - \frac{100}{19} = 0 \\
    c_7 = \sum_{i=0}^{4}(\frac{i^7}{7!} \alpha_i - \frac{i^6}{6!} \beta_i) = 3.0682 - 3.0772 = -0.0090
\end{eqnarray}

From the aforementioned results, it is evident that all coefficients \(c_0, c_1, c_2, c_3, c_4, c_5, c_6\) are found to be zero, while \(c_7\) evaluates to \(-0.0090\). This analysis reveals that Quade's method exhibits a sixth-order convergence and possesses an error constant of \(-0.0090\). Notably, these findings corroborate those reported in the study by \cite{Fadugba2018}.


It scheme is also consistent since \(c_0 = 0 \textbf{ and } c_1 = 0\).
The characteristics equation of the scheme is
\begin{equation}
    \lambda^4 - \frac{8}{19}\lambda^3 + \frac{8}{19}\lambda - 1  = 0
\end{equation}
\begin{equation}
    19x^4 - 8x^3 + 8x - 19 = 0
\end{equation}
\begin{equation}
    x = -1, \quad x = 1, \quad x = \frac{4}{19} + i\frac{\sqrt{345}}{19}, \quad x = \frac{4}{19} - i\frac{\sqrt{345}}{19}
\end{equation}

For \( x = \frac{4}{19} + i\frac{\sqrt{345}}{19} \):
\[
|x| = \sqrt{\left(\frac{4}{19}\right)^2 + \left(\frac{\sqrt{345}}{19}\right)^2} = \sqrt{\frac{16}{361} + \frac{345}{361}} = \sqrt{\frac{361}{361}} = 1
\]

For \( x = \frac{4}{19} - i\frac{\sqrt{345}}{19} \):
\[
|x| = \sqrt{\left(\frac{4}{19}\right)^2 + \left(-\frac{\sqrt{345}}{19}\right)^2} = \sqrt{\frac{16}{361} + \frac{345}{361}} = \sqrt{\frac{361}{361}} = 1
\]

In this context, some of the roots are complex. Zero-stability necessitates that the absolute values have magnitudes less than or equal to 1. Consequently, we affirm that the method demonstrates \textbf{zero stability}.



\begin{figure}[htbp]
    \centering
    \includegraphics[width=1\textwidth]{chapters/4/image/1.png}
    \caption{$\alpha$ $\beta$ - value collector}
\end{figure}

\begin{figure}[htbp]
    \centering
    \includegraphics[width=1\textwidth]{chapters/4/image/2.png}
    \caption{Result of Quade's method analysis}
\end{figure}

\newpage

\subsection{3-Step Backward Differentiation Formula}

The 3-Step Backward Differentiation Formula is given as:

\[y_{n+3} = \frac{18}{11} y_{n+2} - \frac{9}{11} y_{n+1} + \frac{2}{11} y_n + \frac{6}{11} h f_{n+3}\]


from the above equation, we can see that the method is a 3-step method,


where:
\[
\begin{aligned}&\alpha_0 &= -\frac{2}{11}, \alpha_1 = \frac{9}{11}, \alpha_2 = -\frac{18}{11}, \alpha_3 = 1 \\
&\beta_0 &= 0, \beta_1 = 0, \beta_2 = 0, \beta_3 = \frac{6}{11}
\end{aligned}
\]

in other to determine the order of the 3-Step BDF's method, we use (3.4), we obtain the following values:

\begin{align}
    c_0 &= \sum_{i=0}^{3} \alpha_i = -\frac{2}{11} + \frac{9}{11} - \frac{18}{11} + 1 = 0 \\
    c_1 &= \sum_{i=0}^{3} (i\alpha_i - \beta_i) = 0 \cdot \alpha_0 + 1 \cdot \alpha_1 + 2 \cdot \alpha_2 + 3 \cdot \alpha_3 - (\beta_0 + \beta_1 + \beta_2 + \beta_3) \nonumber \\
    &= 0 + 1 \cdot \frac{9}{11} + 2 \cdot \left(-\frac{18}{11}\right) + 3 \cdot 1 - \left(0 + 0 + 0 + \frac{6}{11}\right) \nonumber \\
    &= \frac{9}{11} - \frac{36}{11} + 3 - \frac{6}{11} = 0 \\
    c_2 &= \sum_{i=0}^{3} \left(\frac{i^2}{2!} \alpha_i - i \beta_i \right) = \left(\frac{0^2}{2!} \alpha_0 - 0 \cdot \beta_0\right) + \left(\frac{1^2}{2!} \alpha_1 - 1 \cdot \beta_1\right) + \left(\frac{2^2}{2!} \alpha_2 - 2 \cdot \beta_2\right) + \left(\frac{3^2}{2!} \alpha_3 - 3 \cdot \beta_3\right) \nonumber \\
    &= 0 + \frac{1}{2} \cdot \frac{9}{11} - 0 + \frac{4}{2} \cdot \left(-\frac{18}{11}\right) - 0 + \frac{9}{2} \cdot 1 - \frac{18}{11} \nonumber \\
    &= \frac{9}{22} - \frac{36}{11} + \frac{9}{2} - \frac{18}{11} = 0 \\
    c_3 &= \sum_{i=0}^{3} \left(\frac{i^3}{3!} \alpha_i - \frac{i^2}{2!} \beta_i \right) = \left(\frac{0^3}{3!} \alpha_0 - \frac{0^2}{2!} \beta_0\right) + \left(\frac{1^3}{3!} \alpha_1 - \frac{1^2}{2!} \beta_1\right) + \left(\frac{2^3}{3!} \alpha_2 - \frac{2^2}{2!} \beta_2\right) + \left(\frac{3^3}{3!} \alpha_3 - \frac{3^2}{2!} \beta_3\right) \nonumber \\
    &= 0 + \frac{1}{6} \cdot \frac{9}{11} - \frac{1}{2} \cdot 0 + \frac{8}{6} \cdot \left(-\frac{18}{11}\right) - \frac{4}{2} \cdot 0 + \frac{27}{6} \cdot 1 - \frac{9}{2} \cdot \frac{6}{11} \nonumber \\
    &= \frac{9}{66} - \frac{144}{66} + \frac{27}{6} - \frac{54}{11} = 0 \\
    c_4 &= \sum_{i=0}^{3} \left(\frac{i^4}{4!} \alpha_i - \frac{i^3}{3!} \beta_i \right) = \left(\frac{0^4}{4!} \alpha_0 - \frac{0^3}{3!} \beta_0 \right) + \left(\frac{1^4}{4!} \alpha_1 - \frac{1^3}{3!} \beta_1 \right) + \left(\frac{2^4}{4!} \alpha_2 - \frac{2^3}{3!} \beta_2 \right) + \left(\frac{3^4}{4!} \alpha_3 - \frac{3^3}{3!} \beta_3 \right) \nonumber \\
    &= 0 + \frac{1}{24} \cdot \frac{9}{11} - \frac{1}{6} \cdot 0 + \frac{16}{24} \cdot \left(-\frac{18}{11}\right) - \frac{8}{6} \cdot 0 + \frac{81}{24} \cdot 1 - \frac{27}{6} \cdot \frac{6}{11} \nonumber \\
    &= \frac{9}{264} - \frac{288}{264} + \frac{81}{24} - \frac{162}{22} = -\frac{3}{22} = - 0.13636364
\end{align}




The characteristic polynomial \(P(z)\) associated with the BDF-3 method is given by:

\begin{equation}
    P(z) = 11z^3 - 18z^2 + 9z - 2  
\end{equation}



Solving the Algebraic Equation 

\begin{eqnarray}
    11z^3 - 18z^2 + 9z - 2 = 0 \\
    11z^3 - 18z^2 + 9z - 2 = (z - 1)(11z^2 - 7z + 2) \\    
    11z^2 - 7z + 2 = 0 \\
    (z - 1) = 0 \\
    z = \frac{7 \pm \sqrt{(-7)^2 - 4(11)(2)}}{2(11)} = \frac{7 \pm \sqrt{49 - 88}}{22} = \frac{7 \pm i\sqrt{39}}{22} 
\end{eqnarray}

Therefore, the roots of the polynomial \(11z^3 - 18z^2 + 9z - 2 = 0\) are:

\begin{eqnarray}
    z = 1 \\
    z = \frac{7 + i\sqrt{39}}{22} \\
    z = \frac{7 - i\sqrt{39}}{22}
\end{eqnarray}
We take the modulus of the roots to check for zero-stability:
\begin{eqnarray}
    |z| = |1| = 1 \\
    |z| = \sqrt{\left(\frac{7}{22}\right)^2 + \left(\frac{\sqrt{39}}{22}\right)^2} = \sqrt{\frac{49}{484} + \frac{39}{484}} = \sqrt{\frac{88}{484}} = \sqrt{\frac{22}{121}} = \frac{\sqrt{22}}{11} < 1 \\
    |z| = \sqrt{\left(\frac{7}{22}\right)^2 + \left(-\frac{\sqrt{39}}{22}\right)^2} = \sqrt{\frac{49}{484} + \frac{39}{484}} = \sqrt{\frac{88}{484}} = \sqrt{\frac{22}{121}} = \frac{\sqrt{22}}{11} < 1
\end{eqnarray}

Using Equation (3.4), we can see that the method is a 3-step method of order 3. The method is consistent since \(c_0 = 0\) and \(c_1 = 0\). Furthermore, the method is zero-stable since the roots of the characteristic equation are all inside the unit circle. Lastly, the method is convergent since the error constant is \(-\frac{3}{22}\) or approximately \(-0.1363634\).

Using the JF-Solver to analyze the 3-step Backward Differentiation Formula (BDF(3)), we obtain the following results as shown below:

\begin{figure}[htbp]
    \centering
    \includegraphics[width=1\textwidth]{chapters/4/image/5.png}
    \caption{$\alpha$ $\beta$ - value collector for 3-step BDF method}
\end{figure}

\begin{figure}[htbp]
    \centering
    \includegraphics[width=1\textwidth]{chapters/4/image/6.png}
    \caption{Results of the 3-step BDF method analysis}
\end{figure}

Additionally, the JF-Solver software, which was developed as part of this research, has been used to analyze the BDF(3) method. The software successfully replicated the theoretical results, providing the same error constant and stability characteristics. Moreover, the JF-Solver demonstrated superior performance in terms of computation speed and efficiency, making it a valuable tool for numerical analysis of differential equations.


%! Module 2: Solving Differential Equations

% \newpage
\section{Module 2: Solving Differential Equations}

\subsection{Explicit Method: Adams-Bashforth Method 3-Step Method}
Consider the problem $f(x,y) = 3x^2y$, with the initial condition $y(0) = 1$, the step-size also given as $h = 0.1$.

The Adams-Bashforth 3-step method is given as:
\begin{equation}
    y_{n+3}  = y_{n+2} + h \left(\frac{23}{12}f_{n+2} - \frac{4}{3}f_{n+1} + \frac{5}{12}f_{n}\right)
\end{equation}

The exact solution to the problem is given as:
\begin{equation}
    y(x) = e^{x^3}
\end{equation}




\begin{table}[htbp]
    \centering
    \begin{tabular}{|c|c|c|c|}
        \hline
        Step $n$ & Adams-Bashforth 3-step Method & JF-Solver & Exact Value \\
        \hline
        $y_0$ & 1.000000 & 1.000000 & 1.000000 \\
        $y_1$ & 1.001000 & 1.001000 & 1.001001 \\
        $y_2$ & 1.008032 & 1.008032 & 1.008032 \\
        $y_3$ & 1.027213 & 1.027213 & 1.027398 \\
        $y_4$ & 1.065494 & 1.065494 & 1.066492 \\
        $y_5$ & 1.131580 & 1.131580 & 1.133148 \\
        $y_6$ & 1.237609 & 1.237609 & 1.241857 \\
        $y_7$ & 1.401946 & 1.401946 & 1.409637 \\
        $y_8$ & 1.654090 & 1.654090 & 1.669859 \\
        $y_9$ & 2.043706 & 2.043706 & 2.073784 \\
        \hline
    \end{tabular}
    \caption{Comparison of Results: Adams-Bashforth 3-step Method, JF-Solver, and Exact Values}
    \label{tab:comparison}
\end{table}


The table illustrates the numerical values obtained at each step $n$. The Adams-Bashforth 3-step method and the JF-Solver results are compared against the exact solution $y = e^{x^3}$.

From the comparison, it is evident that both the Adams-Bashforth 3-step method and the JF-Solver produce results that closely approximate the exact solution. The JF-Solver results are rounded to six decimal places, showing a high degree of accuracy.

This comparison demonstrates the effectiveness, accuracy, and reliability of the JF-Solver in approximating the solution of explicit differential equations using the Linear Multistep Method (LMM) given. The application works by accepting the coefficients of the LMM and then solves the differential equation using these coefficients.








